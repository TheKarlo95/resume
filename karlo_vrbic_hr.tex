\documentclass[theme]{cv_style}
\usepackage[utf8]{inputenc}
\usepackage[default]{raleway}
\usepackage{xcolor}
\usepackage[a4paper, portrait, margin=0cm]{geometry}
\usepackage{fontawesome}
\usepackage{array}
\usepackage{enumitem}
\usepackage[pdftex, pdfauthor={Karlo Vrbić}, pdftitle={Karlo Vrbić, CV}, pdfsubject={žIvotopis Karla Vrbića},
pdfkeywords={CV, Resume, Software, Engineer, Android, Tech}]
{hyperref}




\begin{document}
%------------------------------------------------------------------ Variables
% The left column contains the goals, summary, skills, etc.
% We define its width w.r.t. the width of the whole page
\newcommand{\lratio}{0.31}
\newlength{\leftcolwidth}
\setlength{\leftcolwidth}{\lratio\textwidth}
% The right column contains the main content, i.e. work experience, education, etc.
\newcommand{\rratio}{0.7}
\newlength{\rightcolwidth}
\setlength{\rightcolwidth}{\rratio\textwidth}
% Space to leave below a section, above the title of the following section
\newlength{\sectionspace}
\setlength{\sectionspace}{1cm}
% Space to leave below an item, above the following item
\newlength{\itemspace}
\setlength{\itemspace}{10pt}
% fbox stuff. You won't need to adjust these. You can safely ignore.
\setlength{\fboxrule}{0pt}
\setlength{\fboxsep}{4pt}
% Shortcuts to have table columns with fixed width AND positionning: [L]eft, [C]enter, [R]ight
\newcolumntype{L}[1]{>{\raggedright\let\newline\\\arraybackslash\hspace{0pt}}m{#1}}
\newcolumntype{C}[1]{>{\centering\let\newline\\\arraybackslash\hspace{0pt}}m{#1}}
\newcolumntype{R}[1]{>{\raggedleft\let\newline\\\arraybackslash\hspace{0pt}}m{#1}}
% Removes the (ugly) box around html links
\hypersetup{hidelinks}
%------------------------------------------------------------------
\title{Karlo Vrbić}
\author{\LaTeX{} Karlo Vrbić}
\date{2023}



    %-------------------------------------------------------------
    %-------------------------------------------------------------
    %-------------------------------------------------------------
    %                       UPPER PART
    %-------------------------------------------------------------
    %-------------------------------------------------------------
    %-------------------------------------------------------------

    %-------------------------------------------------------------
    %                       HEADER
    %-------------------------------------------------------------
    % Usage: \header{background-color}{name-color}{name}{title-color}{title}{summary-color}{summary}{portrait.jpg}{email@example.com}{phone}{country-flag.png}{city}{linkedin-id}
    \header
    {Karlo Vrbić}
    {Sveučilišni Prvostupnik Inženjer Računarstva $\cdot$ Senior Android\\Developer}
    {
        Kada ne pišem visokokvalitetni kod, volim čitati i učiti o novim događanjima u Androidu\\
        i svijetu softvera općenito, educirati moje kolege, stvarati prezentacije, raspravljati i\\
        poboljšavati razvojne procese te sveukupno razveseliti i motivirati moje suradnike da rade\\
        bolje i da se pritom svi dobro zabavljamo.% Do NOT end with a newline
    }
    {assets/portrait.jpg}

    %-------------------------------------------------------------
    %                       CONTACT BAND
    %-------------------------------------------------------------
    % Usage: \contactband{background-color}{text-color}{email}{phone-number}{country-flag}{city}{linkedin-id}
    \contactband{assets/flag-croatia.png}{Zagreb}{vrbic.karlo@gmail.com}{+385958569156}{karlo-vrbic}{TheKarlo95}{karlo-vrbic}

    \vspace{\headerheight} % The header is only a TIKZ image. We must give it space to appear and not be hidden by what comes next.

    \setlength{\columnsep}{0px}
    \columnratio{\lratio}
    \begin{paracol}{2}
        \paracolbackgroundoptions
        %-------------------------------------------------------------
        %-------------------------------------------------------------
        %                       LEFT COLUMN
        %-------------------------------------------------------------
        %-------------------------------------------------------------
        \begin{leftcolumn*} \noindent \footnotesize
            {\color{white}
            %-------------------------------------------------------------
            %                       GOALS
            %-------------------------------------------------------------
            \heading{\faCompass}{Ciljevi}
            \begin{minipage}[r]{\leftcolwidth}
                \goal{\faAndroid}{Tražim prilike gdje mogu raditi zabavne projekte, proširiti svoje znanje, poboljšati razvojni proces i okruženje u kojem svi dionici uživaju u procesu stvaranja novih proizvoda.}
                \vspace{\itemspace}\\
                \goal{\faBook}{Strastveno dijelim svoje znanje i iskustvo sa svojim suradnicima. Također volim učiti od njih i biti izazovan njihovim novim idejama.}
                \vspace{\itemspace}\\
                \goal{\faLink}{Uživam u povezivanju s ljudima, razmjeni ideja, različitim perspektivama i koristim svaku priliku za profesionalni razvoj.}
                \vspace{\itemspace}\\
                \goal{\faHourglassHalf}{Volim imati vremena da pažljivo razmislim o svojim idejama i dovoljno dobro izvršim zadatke.}
            \end{minipage}

            %-------------------------------------------------------------
            %                       SKILLS
            %-------------------------------------------------------------
            \vspace{1.75\sectionspace}
            \heading{\faPuzzlePiece}{Vještine}
            \begin{minipage}[c]{\leftcolwidth}
                \begin{tabular}{c}
                    \hspace{-3pt}\bubblediagram{
                    % Usage: \bubblediagram{list of comma-separated text items}
                    % The first item will be written in the main bubble, at the center of the diagram
                    % All other items will be written in their own satellite bubble
                        % Main bubble
                        {\textbf{Android} \\ \textbf{Development}},
                        % Satellites
                        Prenošenje\\znanja,
                        Inžinjering,
                        Istraživanje,
                        Kolaboracija,
                        Rješavanje\\problema,
                        Agile\\developemnt}
                \end{tabular}
            \end{minipage}
        }
        \end{leftcolumn*}
        %-------------------------------------------------------------
        %-------------------------------------------------------------
        %                       RIGHT COLUMN
        %-------------------------------------------------------------
        %-------------------------------------------------------------
        \begin{rightcolumn}\noindent \small
            %-------------------------------------------------------------
            %                       WORK EXPERIENCE
            %-------------------------------------------------------------
            \hspace{-2.4pt}\heading{\faSuitcase}{Radno iskustvo}
            % CINNAMON
            \href{https://www.cinnamon.agency}{
            \cvevent{Ožu 2023}{Trenutno}{Senior Android Developer}{Cinnamon}{Zagreb, Hrvatska}{assets/logo-cinnamon.jpg}
            {Rad u digitalnoj agenciji na Android aplikaciji sa BLE tehnologijom. Razvoj novih featurea, održavanje aplikacije i bliska suradnja s drugim članovima tima kako bi se osiguralo da aplikacija zadovoljava potrebe korisnika i da se integrira s ostatkom sustava bile su samo neke od mojih odgovornosti.}}
            \vspace{\itemspace}\\
            % AZIKUS
            \href{https://www.azikus.com}{
            \cvevent{Srp 2020}{Ožu 2023}{Senior Android Developer}{Azikus}{Zagreb, Hrvatska}{assets/logo-azikus.png}
            {Razvijajnje, planiranje i bliska suradnja sa svim dionicima kako bi se izradile mobilne aplikacije i poboljšali proces razvoja unutar firme. Mentoriranje i educiranje novih kolega.}}
            \vspace{\itemspace}\\
            % OPENDYNAMIC
            \href{https://opendynamic.de/}{
            \cvevent{Ruj 2019}{Srp 2020}{Android Developer}{OpenDynamic}{Remote $\cdot$ Mannheim, Germany}{assets/logo-opendynamic.png}
            {Planiranje, dizajn i razvoj mobilne ERP Android aplikacije za skeniranje i praćenje skladišnih naloga na honorarnoj bazi.}}
            \vspace{\itemspace}\\
            % UNDABOT
            \href{https://undabot.com/}{
            \cvevent{Ruj 2018}{Pro 2019}{Android Developer}{Undabot}{Zagreb, Hrvatska}{assets/logo-undabot.png}
            {Razvoj Android aplikacija kao što su A1 mobilna aplikacija, FOREO aplikacija za online kupovinu i mnoge druge. Razvoj messenger library-a za brzu integraciju u druge aplikacije kojima je potrebna mogućnost chat-a.}}
            \vspace{\itemspace}\\
            % COMBIS
            \href{https://www.combis.hr/}{
            \cvevent{Svi 2017}{Ruj 2018}{Mobile Cross-Platform Developer}{Combis}{Zagreb, Hrvatska}{assets/logo-combis.jpg}
            {Razvijanje cross-platform aplikacija s Ionic, Cordove i Angular framework-om te razvoj nativnih Android aplikacija s Kotlinom. Kratki rad na održavanju Java back-end applikacije razvijene sa Strut framework-om.}}
            \vspace{\itemspace}\\
            % ASSECO SEE
            \href{https://see.asseco.com/}{
            \cvevent{Sij 2017}{Tra 2017}{Android Developer}{Asseco SEE}{Zagreb, Hrvatska}{assets/logo-asseco-see.jpg}
            {Razvijanje novih funkcionalnosti i popravljanje bugova na Android mobilnoj aplikaciji za Unicredit banku. U razvoju aplikacije korišten je programski jezik Java i projekt je sastavljen od više modula, flavora i library-a za funkcionalnosti poput skeniranja.}}
            \vspace{0.1cm}\\
            \fbox{
                Radno iskustvo prije siječnja 2017. godine nije softverske prirode ali je vidljivo na
            }%\fbox
            \\
            \fbox{
                \href{https://www.linkedin.com/in/karlo-vrbic}{\faLinkedinSquare \ \textbf{LinkedIn-u}}.
            }%\fbox
            \vspace{0.2cm}\\
        \end{rightcolumn}
        %-------------------------------------------------------------
        %-------------------------------------------------------------
        %                       LEFT COLUMN
        %-------------------------------------------------------------
        %-------------------------------------------------------------
        \begin{leftcolumn*}\noindent \footnotesize
        {\color{white}
            %-------------------------------------------------------------
            %                       TECH
            %-------------------------------------------------------------
            \heading{\faWrench}{Tehnologija}
            \begin{minipage}[c]{\leftcolwidth}
                \begin{tabular}{r|l}
                    Agile development & \pictofraction{3}\\[0.3em]
                    Bluetooth Low Energy & \pictofraction{2}\\[0.3em]
                    Komunikacija & \pictofraction{3}\\[0.3em]
                    Oblikovni obrasci & \pictofraction{4}\\[0.3em]
                    Git & \pictofraction{4}\\[0.3em]
                    Java & \pictofraction{5}\\[0.3em]
                    Jetpack Compose & \pictofraction{3}\\[0.3em]  
                    Kotlin & \pictofraction{5}\\[0.3em]
                    OOP Principi & \pictofraction{4}\\[0.3em]
                    Vođenje tima & \pictofraction{2}\\[0.3em]
                    Testiranje & \pictofraction{2}
                \end{tabular}
            \end{minipage}
        }
        \end{leftcolumn*}
        %-------------------------------------------------------------
        %-------------------------------------------------------------
        %                       RIGHT COLUMN
        %-------------------------------------------------------------
        %-------------------------------------------------------------
        \begin{rightcolumn}\noindent \small
            %-------------------------------------------------------------
            %                       STRENGTHS
            %-------------------------------------------------------------
            \hspace{-2.4pt}\heading{\faHeartbeat}{Interesi \& Stručnost}
            \fbox{
                \begin{minipage}[r]{0.84\rightcolwidth}
                    \cvkeyword{Agile Development}
                    \cvkeyword{Android Development}
                    \cvkeyword{Gradle}
                    \cvkeyword{Clean Arhitektura}
                    \cvkeyword{Decompose/RIBs Arhitektura}
                    \cvkeyword{Kotlin Multiplatform}
                    \cvkeyword{Projekt Managment}
                    \cvkeyword{Team Lead}
                    \cvkeyword{Tech Lead}
                \end{minipage}
            }%\fbox
        \end{rightcolumn}
        %-------------------------------------------------------------
        %-------------------------------------------------------------
        %-------------------------------------------------------------
        %                       PAGE 2
        %-------------------------------------------------------------
        %-------------------------------------------------------------
        %-------------------------------------------------------------
        \newpage
        %-------------------------------------------------------------
        %-------------------------------------------------------------
        %                       LEFT COLUMN
        %-------------------------------------------------------------
        %-------------------------------------------------------------
        \begin{leftcolumn*} \noindent \footnotesize
        {\color{white}
            %-------------------------------------------------------------
            %                       LANGUAGES
            %-------------------------------------------------------------
            \phantom{} \\ % To leave a margin with the top of the page
            \heading{\faGlobe}{Jezici}
            \begin{minipage}[r]{\leftcolwidth}
                \begin{tabular}{r|l}
                    Engleski & Radno znanje\\[0.3em]
                    Hrvatski & Materinji jezik
                \end{tabular}
            \end{minipage}
            \vspace{\sectionspace}
        }
        \end{leftcolumn*}
        %-------------------------------------------------------------
        %-------------------------------------------------------------
        %                       RIGHT COLUMN
        %-------------------------------------------------------------
        %-------------------------------------------------------------
        \begin{rightcolumn}\noindent \small
            %-------------------------------------------------------------
            %                     FORMAL-EDUCATION
            %-------------------------------------------------------------
            \phantom{} \\ % To leave a margin with the top of the page
            \heading{\faGraduationCap}{Formalna Edukacija}
            % UNIVERSITY OF ZAGREB - MASTER
            \href{https://www.fer.unizg.hr/}{
            \cvevent{Ruj 2018}{Srp 2020}{Magistar Inženjer Računarstva}{Sveučilište u Zagrebu, Fakultet Elektrotehnike i Računarstva}{Zagreb, Hrvatska}{assets/logo-unizg.jpg}
            {Fakultet elektrotehnike i računarstva (FER) Sveučilišta u Zagrebu najveći je tehnički fakultet u Hrvatskoj i vodeća obrazovna i istraživačka institucija u područjima elektrotehnike, računarstva i informacijsko-komunikacijske tehnologije. Nudi preddiplomske, diplomske i doktorske programe u širokom rasponu disciplina.}}
            \vspace{\itemspace}\\
            % UNIVERSITY OF ZAGREB - BACHELOR
            \href{https://www.fer.unizg.hr/}{
            \cvevent{Ruj 2013}{Srp 2018}{Sveučilišni Prvostupnik Inženjer Računarstva}{Sveučilište u Zagrebu, Fakultet Elektrotehnike i Računarstva}{Zagreb, Hrvatska}{assets/logo-unizg.jpg}
            {Fakultet elektrotehnike i računarstva (FER) Sveučilišta u Zagrebu najveći je tehnički fakultet u Hrvatskoj i vodeća obrazovna i istraživačka institucija u područjima elektrotehnike, računarstva i informacijsko-komunikacijske tehnologije. Nudi preddiplomske, diplomske i doktorske programe u širokom rasponu disciplina.}}
            \vspace{\itemspace}\\
            % TSRB
            \href{https://www.tsrb.hr/}{
            \cvevent{Ruj 2009}{Svi 2013}{Computer Technician}{Tehnička Škola Ruđera Boškovića}{Zagreb, Hrvatska}{assets/logo-tsrb.png}
            {Tehnička škola Ruđer Bošković je tehnička škola u Zagrebu, Hrvatska. Osnovana je 1948. godine i jedna je od najstarijih i najprestižnijih tehničkih škola u zemlji. Škola nudi niz programa iz elektrotehnike, informatike i drugih tehničkih područja.}}
            \vspace{\itemspace}\\
        \vspace{\sectionspace}
        \end{rightcolumn}
        %-------------------------------------------------------------
        %-------------------------------------------------------------
        %                       LEFT COLUMN
        %-------------------------------------------------------------
        %-------------------------------------------------------------
        \begin{leftcolumn*}\noindent \footnotesize
        {\color{white}
            %-------------------------------------------------------------
            %                       PHILOSOPHY
            %-------------------------------------------------------------
            \heading{\faQuoteLeft}{Filozofija}
            \fbox{
                \begin{minipage}[l]{0.9\leftcolwidth}
                    Evo nekih misli koje vode moje\\
                    aktivnosti kao softverskog inženjera i Android developera.\\[1em]
                    \simplequote{Any fool can write code that a computer can understand. Good programmers write code that humans can understand.}{Martin Fowler}
                    \vspace{\itemspace}\\
                    \simplequote{Good programmers use their brains, but good guidelines save us having to think out every case.}{Francis Glassborow}
                    \vspace{\itemspace}\\
                    \simplequote{Software and cathedrals are much the same; first we build them, then we pray.}{Anonymous}
                    \vspace{\itemspace}\\
                    \simplequote{Android development is a challenging but rewarding career. If you're passionate about building great software, then Android development is a great option for you.}{Jake Wharton}
                    \vspace{\itemspace}\\
                    \simplequote{\faAsterisk Grrhh \faAsterisk \\Tko je ovo napisao?\\ \faAsterisk pogledam git blame \faAsterisk \\ \faAsterisk skužim da sam ja prije par mjeseci \faAsterisk \\ U redu je, valjda \faSmileO}{Ja (ha ha)}
                \end{minipage}
            }%\fbox
        } % \color{white}
        \end{leftcolumn*}
        %-------------------------------------------------------------
        %-------------------------------------------------------------
        %                       RIGHT COLUMN
        %-------------------------------------------------------------
        %-------------------------------------------------------------
        \begin{rightcolumn}\noindent \small
            %-------------------------------------------------------------
            %                       SELF-EDUCATION
            %-------------------------------------------------------------
            \hspace{-2.4pt}\heading{\faTv}{Neformalna Edukacija}
            % INFINUM ACADEMY
            \href{https://infinum.academy/}{
            \cvevent{Srp 2016}{Kol 2016}{Android development}{Infinum Academy}{Zagreb, Hrvatska}{assets/logo-infinum.jpg}
            {Infinum Academy je obrazovni program koji teoriju pretvara u praksu. Kroz tjedne predavanja, zadataka i mentorstva 1-na-1, sudionici razvijaju vještine relevantne za industriju i uče o izradi funkcionalnih digitalnih proizvoda, cijelo vrijeme vođeni od strane priznatih profesionalaca iz industrije.}}

            %-------------------------------------------------------------
            %                       PUBLICATIONS
            %-------------------------------------------------------------
            \vspace{\sectionspace}
            \heading{\faBook}{Publications}
            % Usage: \publication{1:date}{2:title}{3:publisher}{4:publisher-logo}{5:text}
            \href{https://medium.com/azikus/android-how-to-handle-taps-247140d196e9}{
            \publication{Sij 2021}{Android — How to handle taps?}{Medium.com}{assets/logo-medium.png}
            {Blog o tome kako rukovati korisničkim klikovima i gestama te spriječiti neke neželjene nuspojave. Provjerite na \faMedium \ \textbf{Medium}.}}
        \end{rightcolumn}
        \vspace{20em}
    \end{paracol}
\end{document}